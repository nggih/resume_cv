%-------------------------
% Resume in Latex
% Author : Sourabh Bajaj
% License : MIT
%------------------------

\documentclass[letterpaper,10pt]{article}

\usepackage{latexsym}
\usepackage[empty]{fullpage}
\usepackage{titlesec}
\usepackage{marvosym}
\usepackage[usenames,dvipsnames]{color}
\usepackage{verbatim}
\usepackage{enumitem}
\usepackage[hidelinks]{hyperref}
\usepackage{fancyhdr}
\usepackage[english]{babel}
\usepackage{tabularx}

\pagestyle{fancy}
\fancyhf{} % clear all header and footer fields
\fancyfoot{}
\renewcommand{\headrulewidth}{0pt}
\renewcommand{\footrulewidth}{0pt}

% Adjust margins
\addtolength{\oddsidemargin}{-0.5in}
\addtolength{\evensidemargin}{-0.5in}
\addtolength{\textwidth}{1in}
\addtolength{\topmargin}{-.5in}
\addtolength{\textheight}{1.0in}

\urlstyle{same}

\raggedbottom
\raggedright
\setlength{\tabcolsep}{0in}

% Sections formatting
\titleformat{\section}{
  \vspace{-3pt}\scshape\raggedright\large
}{}{0em}{}[\color{black}\titlerule \vspace{-5pt}]

%-------------------------
% Custom commands
\newcommand{\resumeItem}[2]{
  \item\small{
    \textbf{#1}{: #2 \vspace{-2pt}}
  }
}

\newcommand{\resumeSubheading}[4]{
  \vspace{-1pt}\item
    \begin{tabular*}{0.97\textwidth}[t]{l@{\extracolsep{\fill}}r}
      \textbf{#1} & #2 \\
      \textit{\small#3} & \textit{\small #4} \\
    \end{tabular*}\vspace{-5pt}
}

\newcommand{\resumeSubItem}[2]{\resumeItem{#1}{#2}\vspace{-4pt}}

\renewcommand{\labelitemii}{$\circ$}

\newcommand{\resumeSubHeadingListStart}{\begin{itemize}[leftmargin=*]}
\newcommand{\resumeSubHeadingListEnd}{\end{itemize}}
\newcommand{\resumeItemListStart}{\begin{itemize}}
\newcommand{\resumeItemListEnd}{\end{itemize}\vspace{-5pt}}

%-------------------------------------------
%%%%%%  CV STARTS HERE  %%%%%%%%%%%%%%%%%%%%%%%%%%%%


\begin{document}

%----------HEADING-----------------
\begin{tabular*}{\textwidth}{l@{\extracolsep{\fill}}r}
    \textbf{\href{http://nggih.github.io/}{\Large Linggih Saputro}} & Email : \href{mailto:linggih.saputro@sci.ui.ac.id}{linggih.saputro@sci.ui.ac.id}\\
    \href{http://nggih.github.io/}{http://nggih.github.io} & Mobile : +62-877-766-2583 \\
\end{tabular*}




%-----------PROJECTS-----------------
\section{Cover Letter}

In the “Cover letter/other notes Section:” Copy and paste your cover letter of ~750 words that answer the following questions in list format:
        What are your primary research interests and why do you think they are important?
        How would participating in the AI Residency Program help you to explore your research interests and achieve your goals?
        Give an example of an open-ended research question or project you’ve worked on. What made it challenging and how did you overcome those challenges? Alternatively, summarize and critique a machine learning paper you have read that you found interesting.
        **The cover letter is mandatory for the program even though it is optional, as noted on the website, for other jobs at Google**
        In the “Education Section:” attach a current or recent unofficial or official transcript in English.
        Under “Degree Status,” select “Now attending” to upload a transcript.
Regardless of your current profession or area of study, be it physics, medicine, philosophy, or computer science, answer the questions: “How does your previous experience influence your interests and intuitions, and how can you leverage your skills to further research at Google?”

        My research interest is in meta-learning. # NON TECHNICAL/HIGHER LEVEL. stick to simple English
Who reads a cover letter? How Technical should it be?

As a first pass, a team of recruiters will read your cover letter along with your resume and transcript. Researchers at Google AI then review applications that pass the first round. Thus, you should write for both a technical and non-technical audience and make sure your cover letter makes sense to someone who isn't an expert in your field. Non-technical family and friends can be fantastic for this – if your cover letter doesn't make sense to them, you might want to add more background or explain the ideas at a higher level, perhaps via analogy. Overall, stick to simple English and short sentences.
What are they looking for?

The Residency is a chance for Google to hire a diverse set of researchers. This means you should embrace skills, talents, and experiences that make you unique. Regardless of your current profession or area of study, be it physics, medicine, philosophy, or computer science, answer the questions: “How does your previous experience influence your interests and intuitions, and how can you leverage your skills to further research at Google?”

Do you have the potential to excel?

Google recruiters need to know that you have the potential to thrive as a Resident. Highlighting past experience in research, especially in science and engineering, is a good way to convey your potential. Recruiters want to see that you have a solid foundation in coding and math. For example, you could talk about classes you've enjoyed, or some problems or projects on which you have worked. Recruiters are not expecting that you've derived a well-cited theorem, or that you have built a data center from scratch. Less-tangible skills are likewise important: do you understand the uncertainty inherent in research, can you clearly explain your ideas and thought process, and can you learn requisite skills quickly?

What will you gain from the Residency?

Since the Residency is only one year long, it's important that you have an idea of what you hope to learn from your year at Google. Do you want to become more experienced with larger infrastructure? Do you want to learn to apply machine learning to your field? Do you want to apply techniques from your field to machine learning? Communicating your goals will both help you understand if the program is a good fit and will also help Google check whether the Residency will allow you to achieve those objectives while building a class of residents who have a diverse set of goals. 

 Are there teams or researchers at Google AI that align with your interests and skills? Maybe you're interested in algorithms for prescribing medications, models for predicting earthquakes, or systems for finding better neural network architectures. Take a look at the Google AI website to explore the range of research and see if there’s something that sparks your fancy.

If you aren’t positive there is a good fit based on projects that you see online, don’t be afraid to still be honest about what you’d like to spend your year working on. There are more projects in Google AI than are public, so a recruiter would have the best idea if there’s a good fit. For example, a fellow resident’s project dealt with electronic health records – details of which, at the time, could not be disclosed. She would not have known about this research happening at Google, but found a good project match with Google Health after joining. Furthermore, a year is not a short amount of time, and if there isn’t a good fit, it would be a shame to spend it working on something that you don’t enjoy.

What will you contribute to Google's culture?

Once application readers have established you’re a great research and technical fit, they want to know that you’ll be a great cultural fit. Will you speak up for things you believe in, technical or otherwise? Will you work together with fellow residents and contribute different perspectives and life experiences to your resident cohort? Will you have your own opinions on what is important to work on, but be willing and able to have discussions with collaborators to scope out projects? Will you attend paper discussion groups and give talks about your research? In our opinion, your ability to contribute to the community at Google AI is at least as important as your technical skills. In seeking to find future leaders in AI, the Residency looks for candidates that will make the machine learning community interesting and welcoming. 

%-------------------------------------------
\end{document}

